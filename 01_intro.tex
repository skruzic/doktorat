\chapter{INTRODUCTION}
\label{chap:Intro}

Robots have become ubiquitous over the past decades, with different areas of application in automated industrial processes and beyond. Thus, it is one of the fastest-growing industries today and a trendy field among researchers where new and innovative robotics applications are developed continually. It is especially highlighted in some recent reports that estimated the global robotics market's value at nearly 28 billion dollars in 2020 \cite{MordorIntelligence2020}. Furthermore, according to the forecasts, the market value is expected to reach about 74 billion dollars by 2026 due to the high demand for robots (especially the automotive industry) and logistics. In addition, however, the demand for service robots, both for professional and domestic use, is also increasing. This high demand for robots of all kinds is due to recent advances in automation and robotics, which enable their usage in demanding applications while simultaneously reducing operating costs.

Robots can be broadly divided into two big groups: mobile robots and robotic manipulators. Mobile robots are characterised by the ability to move in space and include ground-based mobile robots, unmanned aerial vehicles (UAV, drones) and remotely operated underwater vehicles (ROV). In recent years, mobile robots started to be found in various non-industrial applications increasingly: ones helping people (termed \emph{assistive robotics}), or ones providing some services (\emph{service robotics}, e.g. home-based autonomous vacuum cleaners) and in numerous variants of mobile manipulators in commercial applications such as in warehouses and stores, but started to come into homes as well. In these cases, the mobile manipulator is defined as a robotic manipulator that is mounted (attached) to the mobile robotic base and can thus move in two-dimensional space.

On the other hand, robotic manipulators are fixed in one place. They perform various tasks and usually work in an open loop, i.e., they are programmed to perform predefined trajectories (movement) to complete a given task and have no external sensors to provide feedback on interaction (and thus close the external control loop). Therefore, it is not possible to know if there was a problem performing the task. However, in applications where the position/configuration of the robot alone cannot guarantee the successful execution of a given task, this approach is inappropriate. These are usually tasks that require physical interaction between the robot and the environment and in the collaboration of man and robot or several different robots (which is often the case in Industry 4.0 scenario). To be able to asses the interaction, the interaction needs to be measured. The interaction in the physical sense consists of forces and torques, which can be measured by mounting a sensor on the robot to measure the forces and torques. Furthermore, the robot usually interacts with the environment by touching the environment with its tip. Thus, the force and torque sensor is most often mounted on the robot tip.

Sometimes it is inconvenient or impossible to mount a force and torque sensor on the robot end-effector (especially emphasised in smaller robots, which are usually intended for education) due to the relatively large mass of such sensors, which significantly reduces the payload of the robot. Therefore, it is necessary to find another way to reliably (and indirectly) measure the interaction. The logical possibility that arises is to place force-torque sensors under the robot base because this does not affect the payload and allows to combine measured values and robot models (i.e., geometry and physics) to estimate force values at the robot's tip (or any other robot joint, if necessary).

The usual way to solve a problem of an indirect assessment of interaction involves the incorporation of a dynamic model of a robot, which must be accurate because otherwise, the obtained results would not be satisfactory \cite{Colome2013}.
The motivation for researching this area lies in applying deep learning methods and neural networks to solve this problem. Namely, neural networks can generalise, so a robot model is unnecessary because it can be learnt implicitly from data. Another advantage of this approach is that the robot model, once known, can be used in real-time even when computing resources are relatively limited, while numerical methods for solving differential or integral equations of robot dynamics on a computer are much slower and contain certain approximations, these may be inappropriate for real-time execution.

In addition to all the above, robotic manipulators have one additional significant disadvantage: they are attached at a fixed place and consequently have a relatively narrow and fixed working environment. Therefore, a logical step toward significantly increasing the working environment may be to place a robotic manipulator on a mobile robotic base forming a mobile robotic manipulator. Such mobile manipulators potentially have numerous applications in modern industry \cite{Madsen2015}, logistics \cite{Iriondo2019} and assistive robotics \cite{Park2020}.

However, the mobility of the robotic manipulator opens up new issues that need to be addressed and related to autonomous navigation. In general, navigation in a broader sense involves autonomous driving to a given destination (or following a given trajectory) and avoiding obstacles (static and dynamic). However, avoiding obstacles cannot always be fully reconciled with navigating to a destination (sometimes they have the opposite effect -- avoiding obstacles can, in some situations, contribute to moving away from a destination or a given route). Therefore, a navigation approach should be developed in which it will be possible to separate these two components into two separate controllers, each for one type of behaviour (driving to the goal and avoiding obstacles), and find a way to combine these two (primary) behaviours to complex behaviour occurred. Furthermore, this approach should separate the navigation and obstacle avoidance logic, allowing each behaviour to be freely and easily changed without affecting the other.

\section{Hypotheses}

In the research, the following hypotheses were formed:

\begin{itemize}
    \item it is possible to develop a method for interaction force estimation at the robot tip using neural networks of appropriate architecture and the measurements from the force sensor mounted under the robot base without knowing the robot's dynamic model explicitly
    \item it is possible to achieve comparable results using multiple single-axis and low-cost strain gauge sensors in place of standard 3-axis force sensors
    \item it is possible to achieve mobile robot navigation in known and unknown environments with obstacle avoidance, based on mediation among multiple basic behaviours of the mobile robot
    \item it is possible to train neural networks for mediation-based navigation and forces and torques estimation entirely using data obtained in simulation without the need for domain adaptation, and \emph{simulation-to-reality gap} can be shrunk using simple data, e.g. 2D LiDAR for navigation
\end{itemize}

\section{Dissertation organisation}

In the introductory chapter, motivations for the research are given, and the research hypotheses are formulated. Then, in \cref{chap:Literature}, some preliminaries related to neural networks are introduced since neural networks were heavily used in the research. Furthermore, a literature review of the state-of-the-art is given, both for fields of mobile robot navigation and force and torques estimation or robotic manipulators. Next, \cref{chap:Materials} formulates research problems, proposes solutions and explains experiments that were conducted to assess the performance of the proposed solutions. Then, in \cref{chap:Results} results obtained are presented, thoroughly analysed and discussed. Finally, \cref{chap:Conclusions} draws conclusions based on the obtained results and states the directions of future research.

\newpage