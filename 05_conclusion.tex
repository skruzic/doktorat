\chapter{CONCLUSIONS}
\label{chap:Conclusions}

The motivation for this research is in the rising use of mobile robotic manipulators, both in domestic and industrial environments. Since the mobile robotic manipulators consist of a robotic manipulator mounted on the mobile base, this dissertation addressed both aspects of the mobile robotic manipulator separately. 

%The idea of fuzzy-based mediation for (simple) mobile robot navigation tasks was presented in this dissertation and implemented on a couple of real robots. It combines the outputs of two controllers, one dedicated to navigation to the goal point and the other for obstacle avoidance, to the final output, which is (in general) a combination of the outputs of the two controllers. The approach is thoroughly tested in simulation and real-world scenarios using two mobile robots of different architecture, footprint sizes, and shapes.

While it was previously shown that neural networks could be used to solve robot navigation tasks and obstacle avoidance tasks separately, the experiments in the research demonstrated that the problem could be decomposed into distinct primitives (with or without the use of neural networks) by applying fuzzy mediation, thus adapting robot behaviour. Also, the approach is not limited only to neural network-based controllers, but other controller types like PID controllers can be used. In this manner, more complex behaviour can be achieved by combining task primitives while enabling modularity. That is, controllers addressing specific tasks can be changed without influencing other task controllers. Furthermore, obtained results demonstrate that including different types of controllers without changing the mediation mechanism is possible, making additional adaptation and optimisation possible. Although results show good performance (especially in practical tasks), comparison to a baseline method shows some differences (in general slower completion time and longer distance travelled, although there were some cases where the proposed approach was on average faster and with shorter trajectory length). However, the overall performance of the proposed mediation scheme is satisfactory, and it does not require any human intervention, demonstrating that it is valid for use in practical applications.

In addition to the mediation, a self-contained method for neural network-based obstacle avoidance was developed. The approach was designed to fit into the proposed fuzzy mediation framework, but it can also work independently. Furthermore, the method is characterised by collecting training data conducted in simulation and using LiDAR as a source of information, contrary to most state-of-the-art applications, which use video streams. While neural networks were proved applicable to obstacle avoidance in mobile robotics, experiments in this dissertation demonstrate that desired behaviour in obstacle avoidance can be achieved based on data collected in simulation in a self-supervised manner with no or very little human intervention. This approach makes collecting an abundance of training data effortless, a task that is very difficult, time-consuming and potentially unsafe for the robot and the environment when doing it in the real world. It was also demonstrated that a high degree of similarity exists between simulation and real-world scenarios and that obstacle avoidance could be achieved by using LiDAR-based data. It resulted in a lower complexity level of the neural network trained on simulation-based data and performed satisfactorily without retraining within real-world environments. Good results were achieved in experiments where the proposed approach was tested in simulation and on the real robot in several complex obstacle avoidance tasks in demanding environments, demonstrating the practical applicability of the proposed approach.

Finally, methods for estimating robot manipulator end-effector forces and joint torques based on neural networks are developed. The knowledge of end-effector forces is essential for any robot operation that requires interaction with the environment (or human or another robot), while the knowledge of torques is vital when a force control scheme is used. In the developed approach to end-effector force estimation, forces are measured with a force sensor mounted under the robot base, and end-effector forces are estimated using deep neural networks. The approach was tested on two robots in simulation and the real world. It was shown that simulation-based robot performed better, which was expected, and that the network for force estimates do not need to be very complex networks (in the number of layers and neurons per layer). It was also proved that providing more input features about the robot state led to better generalisation and performance. However, the real-world performance of the proposed method for one of the robots was not adequate, but it was concluded that it was due to the faulty end-effector-mounted sensor. On the other side, the proposed method for joint torques estimation performed adequately in all cases, both in simulation and the real world. Furthermore, a modification of the approach was experimented with, where four low-cost, single-axis sensors replaced base-mounted force sensors. This modification performed adequately, but slightly worse than the original setup with a force sensor, but still demonstrating its usability, despite less reliable sensors. Although both proposed force and joint torque estimation methods (and their variation) are developed with mobile manipulators in mind, they can also be used with stationary robotic manipulators.

The simulation was an essential part of the presented research. It was used to generate data for training neural networks (for obstacle avoidance, navigation to a given goal and forces and torques estimation). It enabled the collection of extensive datasets for training and testing neural networks in a safe and controlled manner. The simulation-to-reality gap was bridged successfully employing simple sensors (providing low complexity data),  thus enabling the use of simulation-trained networks on real robots without retraining with real-world data. During the research, the simulation was running for four days total, in a self-supervised manner, collecting data without human intervention once started.

From the conducted experiments and their results and discussion, the contributions of the dissertation are as stated as follows:

\begin{itemize}
    \item The self-supervised neural network for mobile robot obstacle avoidance was developed. The inputs to the network are LiDAR sensor data, and the network was trained using only data generated in simulation. This method was made applicable to the real-world robot directly, without any additional training using real-world data.
    \item The method for mobile robot navigation in non-structured environments was developed. The method achieves complex behaviour by mediation (based on fuzzy logic) between the controllers in charge of one primitive behaviour. The method can be used as a basis for integrating a mobile base with a robotic manipulator.
    \item Neural network-based methods for the end-effector force and joint torques estimation of a robotic manipulator were developed. Methods estimate interaction forces (or joint torques) acting on the manipulator tip based on the measurements of the sensor mounted under the robot base. The variation of the approach where the base-mounted force sensor is replaced with four low-cost single-axis force sensors was also developed. It was based on strain gauges and demonstrated results comparable to the previous.
\end{itemize}


Based on the obtained results and the discussed advantages and disadvantages of the methods presented in this thesis, the directions of the future research are the following:


\begin{itemize}
    \item The integration of the mobile platform and robotic manipulator and its control as a whole
    \item Make obstacle avoidance algorithm more adaptive so that it can get out of some difficult obstacle configurations, like convex dead ends
    \item Assess the proposed force and torque estimation approach on different real-world mobile manipulators
    \item Improve the simulation-based data collection procedure in simulation so that no preprocessing of raw generated data is needed.
\end{itemize}



\newpage