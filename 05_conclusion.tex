\chapter{CONCLUSIONS}
\label{chap:Conclusions}

The motivation for this research was in the rising use of mobile robotic manipulators, both in domestic and industrial environments. Since the mobile robotic manipulators consist of a robotic manipulator mounted on the mobile base, this dissertation addressed both aspects of the mobile robotic manipulator. 

The idea of fuzzy-based mediation for (simple) mobile robot navigation tasks was presented in this dissertation and implemented on a couple of real robots. It combines the outputs of two controllers, one dedicated to navigation to the goal point and the other for obstacle avoidance, to the final output, which is (in general) a combination of the outputs of the two controllers. The approach is thoroughly tested in simulation and real-world scenarios using two mobile robots of different architecture, footprint sizes, and shapes.

While it was previously shown that neural networks could be used to solve robot navigation tasks and obstacle avoidance tasks separately, the experiments demonstrated that the problem could be broken down into task primitives (with or without the use of neural networks) with the application of adaptive fuzzy mediation, thus adapting robot behaviour. The neural network ensembles approach for fusing the output of two or more neural networks has been demonstrated before, but the proposed approach enables a more transparent manner in which this is done and can be fine-tuned intuitively to change adaptively. Also, the approach is not limited only to neural network-based controllers, but other controller types like PID controllers can be used. In this manner, more complex behaviour can be achieved by combining task primitives while enabling modularity. That is, controllers addressing specific tasks can be changed without influencing other task controllers. Furthermore, obtained results demonstrate that including different types of controllers (while keeping the mediation mechanism unchanged) is possible, which opens up additional possibilities for adaptation and optimisation. Although results show good performance (especially in practical tasks), comparison to a baseline method shows some differences (in general slower completion time and longer distance travelled, although there were some cases where the proposed approach was on average faster and with shorter trajectory length). However, overall performance is satisfactory (and without any human intervention). 

A self-contained approach for safe and accurate generation of neural network-based training data and such data for the mobile robot obstacle avoidance was developed. It was designed in a way to fit the fuzzy mediation framework. While it was previously shown that neural networks could be used for obstacle avoidance in mobile robotics, experiments in this paper demonstrate that it can be done based on the simulated data in a self-supervised manner with no or very little human intervention. This approach, in turn, makes collecting substantial amounts of training data more effortless, a task that is very difficult and time-consuming using a real robot. The paper also demonstrated that a high degree of similarity between simulation and real-world scenarios could be achieved by using LiDAR-based data instead of more commonly used live camera images/feed. Also, it resulted in a lower complexity level of the neural network trained on simulation-based data and performed satisfactorily without retraining within real-world environments. Good results were achieved in experiments where the proposed approach was tested in simulation and on the real robot in several complex obstacle avoidance tasks in demanding environments, demonstrating the practical applicability of the proposed approach.

The method for estimation of robot manipulator end-effector forces is developed, with the robotic manipulators in mind. Forces are essential for any robot operation that requires interaction with the environment. In the developed approach, forces are measured with a force sensor mounted under the robot base, and end-effector forces are estimated using deep neural networks. The approach was tested on two robots in simulation and the real world. It was shown that simulation-based robot performed better, which was expected, and that the network for force estimates do not need to be very complex networks (in the number of layers and neurons per layer). Furthermore, providing more input features about the robot state led to better generalisation.

\section{Contributions}

The contributions of the dissertation are as follows:

\begin{itemize}
    \item The self-supervised neural network for mobile robot obstacle avoidance was developed. The inputs to the network are LiDAR sensor data, and the network was trained using only data generated in simulation. This method was made applicable to the real-world robot directly, without any additional training using real-world data.
    \item The method for mobile robot navigation in non-structured environments was developed. The method achieves complex behaviour by mediation (based on fuzzy logic) between the controllers in charge of one simple behaviour. The method can be used as a basis for integrating a mobile base with a robotic manipulator.
    \item The method based on neural networks for the estimation of robotic manipulator interaction forces was developed. The method does the estimation of forces acting on the manipulator tip based on the measurements of the sensor mounted under the robot base.
\end{itemize}

\section{Further research}

Based on the obtained results and its discussion in this thesis, the directions of the further research are the following:

\begin{itemize}
    \item The integration of the mobile platform and robotic manipulator and its control as a whole
    \item Make obstacle avoidance algorithm more adaptive so that it can get out of some difficult obstacle configurations, like convex dead ends
    \item Assess the proposed force estimation approach on different real-world robotic manipulators
    \item Force estimation based on the measurements of multiple 1D strain gauges under the robot base instead of the force sensor 
\end{itemize}



\newpage