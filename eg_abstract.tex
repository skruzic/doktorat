% -------------------------------------------------------------------
% Author: Toni Perković, toperkov@fesb.hr, toperkov@unist.hr
% Author: Marin Bugarić, mbugaric@fesb.hr
% Author: Ivo Stančić, istancic@fesb.hr
% FESB 2016;
% -------------------------------------------------------------------

%abstract in Croatian

\newpage
\setlength{\parindent}{0in}
{\fontsize{14}{18}\bf {Metode umjetne inteligencije za procjenu sila interakcije i medijacijsku navigaciju robotskog manipulatora na mobilnoj platformi}}

\vskip 15mm
% \section {SA\v{Z}ETAK}
\addcontentsline{toc}{section}{Sa\v{z}etak}
% \section*{SA\v{Z}ETAK}
\begin{otherlanguage}{croatian}
\textbf{Sa\v{z}etak:\\}
   
\textnormal{Mobilni robotski manipulatori su se pojavili tek nedavno, a već ih se može naći u industrijskim okolinama i domovima. Mogu obavljati razne zadatke zbog svoje mogućnosti kretanja u prostoru. Budući da se sastoje od mobilne baze na koju je montiran standardni robotski manipulator, potrebno je znanje iz područja mobilnih robota i robotskih manipulatora da bi se efikasno upravljalo takvim složenim sustavom. Stoga je potrebno razviti efikasne upravljačke sheme za mobilne platforme i za robotske manipulatore. Ova disertacija obuhvaća oba polja i predlaže rješenja za česte probleme u njima. Prvo je razvijena efikasna i modularna upravljačka shema koja ostvaruje kompleksno ponašanje mobilnog robota. To se postiže korištenjem medijacije temeljene na neizrazitoj logici, u cilju fuzije dva jednostavna ponašanja (navigacija i izbjegavanje prepreka). Također je razvijena i metoda za izbjegavanje prepreka temeljena na neuronskim mrežama, te uspješno ugrađena u shemu medijacije. U provedenim eksperimentima, dva stvarna robota različitih veličina i oblika su pokazala dobre rezultate. Nadalje, predložen je i razvijen i pristup za procjenu sila koje djeluju na vrh robotskog manipulatora, te momenata u zglobovima robota. Pristupi se temelje na neuronskim mrežama, te koriste mjerenje senzora koji je postavljen ispod baze robota. Mreže su trenirane za dva manipulatora različitih veličina i nosivosti. One dobro generaliziraju za nepoznate trajektorije, te su procjene sila na vrhu i momenata u zglobovima prilično točne.}

\vskip 15mm
\bf{Klju\v{c}ne rije\v{c}i:\\}
\textnormal{mobilni manipulator, mobilni robot, robotski manipulator, neuronske mreže, duboko učenje, medijacija, neizrazita logika, navigacija, izbjegavanje prepreka, procjena sila}
\end{otherlanguage}