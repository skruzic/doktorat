% -------------------------------------------------------------------
% Author: Toni Perković, toperkov@fesb.hr, toperkov@unist.hr
% Author: Marin Bugarić, mbugaric@fesb.hr
% Author: Ivo Stančić, istancic@fesb.hr
% FESB 2016;
% -------------------------------------------------------------------

%\newpage \mbox{} \newpage \pagestyle{empty}

\newpage \mbox{} \newpage \pagestyle{empty}

%\begin{flushleft}
\section*{Curriculum Vitae}

\vspace{15mm}

{\noindent\bf{Stanko Kružić}} was born on 11\textsuperscript{th} August 1985 in Split, Croatia. After completing the Electrical Engineering graduate programme in 2009 at FESB, he was awarded a Master of Electrical Engineering degree. Following his studies, he worked at the University of Split (2011-2016), University Department of Health Studies, as Head of IT office. In 2015, he applied for the PhD programme in Electrical Engineering and Information Technology at FESB and, in the following year, applied for the position of Research assistant at FESB, Department of Electronics and Computer Science, where he is still working today. His research interests is in the field of robotics and applying deep learning models to tackle various tasks in mobile robots and mobile manipulators. 

\noindent Stanko has been a member of two research projects: Increasing the Well Being of the Population by Robotic and ICT Based Innovative Education (RONNI) with project partners from Institute of Robotics, Sofia, Bulgaria and Eastern Macedonia and Thrace Institute of Technology, Kavala, Greece; and Smartbots - Autonomous Control of Mobile Robots Using Computer Vision Algorithms and Modern Neural Network Architectures with project partners from Bonn-Rhein-Sieg University of Applied Sciences, Sankt Augustin, Germany. He has attended three summer schools: Summer School on Computational Interaction in Luzern, Switzerland (2017), Thematic CERN School of Computing in Split, Croatia in (2018), and Eastern European Machine Learning Summer School in Bucharest, Romania (2019), and an Erasmus+ Staff Mobility in Ljubljanja, Slovenia (2019).

\noindent He has published six journal papers (four of which are WoS-indexed) and seven conference papers.


%\end{flushleft}
\newpage \pagestyle{empty}

%HR ŽIVOTOPIS
\section*{Životopis}

\vspace{15mm}

{\noindent\bf{Stanko Kružić}} je rođenn 11. kolvoza 1985. u Splitu. Nakon završetka dodiplomskog studija Elektrotehike 2009. godine na FESB-u, dodijeljeno mu je zvanje diplomirani inženjer elektrotehnike. Nakon studija, radio je na Sveučilištu u Splitu (2011-2016), Sveučilišni odjel zdravstvenih studija, kao voditelj IT službe. Godine 2015. upisuje doktorski studij Elektrotehnike i informacijske tehnologije na FESB-u, te u sljedeće godine dobiva poziciju asistenta na FESB-u, Zavod za elektroniku i računarstvo, gdje radi i danas. Njegov istraživački interes je u polju robotike i aplikacijama modela dubokog učenja za izvršavanje različitih zadataka kod mobilnih robota i mobilnih manipulatora. 

\noindent Stanko je bio član dva istraživačka projekta: ``Increasing the Well Being of the Population by Robotic and ICT Based Innovative Education (RONNI)'' s projektnim partnerima s Instituta za robotiku, Sofija, Bugarska and Tehnnološki institut Istočne Makedonije i Trakije, Kavala, Greece; i ``Smartbots - Autonomous Control of Mobile Robots Using Computer Vision Algorithms and Modern Neural Network Architectures'' s projektnim partnerima sa Sveučilišta primjenjenih znanosti Bonn-Rhein-Sieg, Sankt Augustin, Njemačka. Prisustvovao je trima ljetnim školama: ``Summer School on Computational Interaction'' u Luzern-u, Švicarska (2017), ``Thematic CERN School of Computing'' u Splitu (2018) i ``Eastern European Machine Learning Summer School'' u Bukureštu, Rumunjska (2019), te mobilnosti iz programa Erasmus+ Staff Mobility u Ljubljanji, Slovenija (2019).

\noindent Autor je šest članaka u znanstvenim časopisima (od čega su četiri indeksirani u WoS-u) i sedam članaka na znanstvenim skupovima.

%\begin{flushleft}
%\section*{\v{Z}ivotopis}

%\vspace{2mm}

%{\noindent\bf{Ime Prezime}\\}

%\vspace{5mm}

%\v{Z}ivotopis autora doktorskog rada treba biti napisan u tre\'{c}em licu jednine, a opsegom ne smije prelaziti 1500 znakova (uklju\v{c}uju\'{c}i razmake).

%\end{flushleft}
\newpage 